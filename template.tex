\usepackage{okumacro} % jsclassesに同梱のパッケージ いろんなマクロがある
% 和文の仮想ボディがアレ
\usepackage{otf}
% 欧文フォントのサイズ指定がアレ
\usepackage[T1]{fontenc}
\usepackage{lmodern}
% 用紙サイズの扱いがアレ
\usepackage{bxpapersize}
% 空白文字を含む画像を読み込めるようにする
\usepackage{grffile}
% 余白設定
\usepackage[top=20mm, bottom=25mm, left=20mm, right=20mm]{geometry}

\usepackage{hyperref} % ハイパーリンクつき文章にする
\usepackage{pxjahyper}
\usepackage{xcolor}
\hypersetup{
    colorlinks=false,
    citebordercolor=green,
    linkbordercolor=red,
    urlbordercolor=cyan,
}

\usepackage{graphicx} %画像読み込み
\usepackage{color} %png画像が表示されないとき
\usepackage{amsmath} %数式
\usepackage{bm} % 太字、ベクトル
\usepackage{textcomp} % 特殊文字
\usepackage{siunitx} % SI単位系
\usepackage{paralist} % 改行しない箇条書きを可能にする 
\usepackage{url} %url
\usepackage{multirow} %表 縦結合
\usepackage{multicol}
\usepackage{subfigure} %副番号
\usepackage{fancyhdr} %ヘッダ フッタ
\usepackage{float}
\usepackage{tikz}
\usepackage[RPvoltages]{circuitikz}
\usepackage{cases}

\usepackage{here} %画像配置
    \pagestyle{fancy}
    \lhead{}
    \chead{}
    \rhead{\thepage}
    \lfoot{}
    \cfoot{}
    \rfoot{}
    \renewcommand{\headrulewidth}{0pt}

\usepackage{minted}
\usepackage{listings,jlisting} %日本語のコメントアウトをする場合jlistingが必要
\usepackage{txfonts}

% PDFの表紙を読み込む
\usepackage{pxpdfpages}
\usepackage{pdfpages}

% solarized
\definecolor{base}{gray}{0} %black
\definecolor{comment}{rgb}{0.52,0.60,0.00} %green
\definecolor{string}{rgb}{0.83,0.21,0.51} %magenta
\definecolor{keyword1}{rgb}{0.15,0.55,0.82} %blue
\definecolor{keyword2}{rgb}{0.80,0.29,0.09} %orange
\definecolor{keyword3}{rgb}{0.71,0.54,0.00} %yellow
\definecolor{keyword4}{rgb}{0.42,0.44,0.77} %violet[f:id:e8l:20151129232557p:plain][f:id:e8l:20151129232557p:plain][f:id:e8l:20151129232557p:plain]

\lstset
{
    basicstyle={\ttfamily\color{base}\scriptsize},%コードの基本書式
    keywordstyle=[1]{\color{keyword1}\textbf},%キーワード1のスタイル
    keywordstyle=[2]{\color{keyword2}\textbf},%キーワード2のスタイル
    keywordstyle=[3]{\color{keyword3}\textbf},%キーワード3のスタイル
    keywordstyle=[4]{\color{keyword4}\textbf},%キーワード4のスタイル
    commentstyle={\gtfamily\scriptsize\color{comment}},%コメントのスタイル
    stringstyle={\gtfamily\scriptsize\color{string}},%文字列のスタイル
    numbers=left,%行番号は左
    stepnumber=1,%一行ずつ行番号をふる
    numberstyle={\sffamily\scriptsize},%行番号の書式
    xleftmargin=0zw, %左余白
    xrightmargin=0zw,%右余白
    tabsize=4,%タブの空白数
    % frame=single,フレームの書式
    frame={tb},
    % frameround=tttt,角を丸めるかどうか tで丸める
    breaklines=true,%長くなったら途中で改行
    captionpos=t,%タイトルの位置
    breakindent=10pt,%改行されたときの送り幅
    showstringspaces=false,%文字列中の半角スペースを表示させない
    lineskip=-1pt%通常の文章より行送りを狭くする
}

% mktitleを調整
\usepackage{mdframed}
\makeatletter
\newcommand{\subtitle}[1]{\def\@subtitle{#1}}
\newcommand{\institute}[1]{\def\@institute{#1}}
\renewcommand{\maketitle}{
  \begin{mdframed}[roundcorner=10pt]
    {\small \@date}
    \vspace{-2.5ex}
    \begin{center}
      \textgt{\@subtitle}\\[0.25ex]
      {\Large \textgt{\@title}}
    \end{center}
    \vspace{-3.5ex}
    \begin{flushright}
      \@institute \hspace{0.5em} \@author
    \end{flushright}
  \end{mdframed}
}
\makeatother

\newcommand{\reffig}[1]{{図 \ref{fig:#1}}}
\newcommand{\reftab}[1]{{表 \ref{tab:#1}}}
\newcommand{\refeq}[1]{式 \eqref{eq:#1}}
\newcommand{\refsec}[1]{\ref{sec:#1}章}
\newcommand{\refsc}[1]{{ソースコード \ref{sc:#1}}}
\newcommand{\refres}[1]{{実行結果 \ref{sc:#1}}}
